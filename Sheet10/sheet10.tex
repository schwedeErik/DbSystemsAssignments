
\documentclass[a4paper]{article}
\usepackage[utf8]{inputenc}

\usepackage[english,german]{babel} 
\usepackage[utf8]{inputenc}

\usepackage{alltt}
\usepackage{amsmath}
\usepackage{amssymb}
\usepackage{amsthm}
\usepackage{color}
\usepackage{enumitem}
\usepackage{epsfig}
\usepackage{fancyhdr}
\usepackage{float}
\usepackage{framed}
\usepackage{graphicx} 
\usepackage{graphics}
\usepackage{hyperref}
\usepackage{listings}
\usepackage{multirow}
\usepackage{tabularx}
\usepackage{textcomp}
\usepackage{tikz}
\usepackage{url}
\usepackage{vmargin}
\usepackage{xspace}
\usepackage{comment}
\usetikzlibrary{calc,trees,positioning,arrows,chains,shapes.geometric,%
    decorations.pathreplacing,decorations.pathmorphing,shapes,%
    matrix,shapes.symbols,topaths,matrix}
\newcommand{\question}[2][0]{\section{{#2} \hfill ({#1} P.)}}
\setpapersize{A4}
\setmargins{2.5cm}{2.0cm}% % linker & oberer Rand
         {16cm}{22cm}%   % Textbreite und -hoehe
           {48pt}{36pt}%   % Kopfzeilenhoehe und -abstand
           {0pt}{30pt}%    % \footheight (egal) und Fusszeilenabstand

\frenchspacing
\pagestyle{fancy}
\sloppy

\newcommand{\hide}[1]{}

\markright{Kopfzeile}


\tikzset{
>=stealth',
  punktchain/.style={
    rectangle,
    rounded corners,
    % fill=black!10,
    draw=black, very thick,
    text width=10em,
    minimum height=3em,
    text centered,
    on chain},
  line/.style={draw, thick, <-},
  element/.style={
    tape,
    top color=white,
    bottom color=blue!50!black!60!,
    minimum width=8em,
    draw=blue!40!black!90, very thick,
    text width=10em,
    minimum height=3.5em,
    text centered,
    on chain},
  every join/.style={->, thick,shorten >=1pt},
  decoration={brace},
  tuborg/.style={decorate},
  tubnode/.style={midway, right=2pt},
}


\setlength{\parindent}{0pt}
\setlength{\parskip}{5pt}
\fboxsep1.5mm


\lstdefinestyle{mystyle}{
    backgroundcolor=\color{white},   
    commentstyle=\color{codegray},
    keywordstyle=\bf \ttfamily \color{codepurple},
    numberstyle=\tiny\color{codegray},
    stringstyle=\color{codegreen},
    basicstyle=\footnotesize,
    breakatwhitespace=false,         
    breaklines=true,                 
    captionpos=b,                    
    keepspaces=true,                 
    numbers=left,                    
    numbersep=5pt,                  
    showspaces=false,                
    showstringspaces=false,
    showtabs=false,                  
    tabsize=8,
    keepspaces,
    extendedchars=true, 
      upquote=true,
    columns=fixed,
    showstringspaces=false,
    extendedchars=true,
    breaklines=true,
    frame=single,
    showspaces=false,
    showstringspaces=false,
    rulecolor=\color{white},
}

\lstdefinelanguage{sql}[]{}{
        %tag=[s]<>,      % =*: also apply styles within tag, =**: cumulate styles
        morekeywords={sql, VIEW, AS, FROM, SELECT, WHERE, FUNCTION, BOOLEAN, RETURNS, DETERMINISTIC, RETURN, REFERENCES, WITH, SEQUENCE, TRUNCATE, START, CREATE, AS, LANGUAGE, FUNCTION, CURSOR, PREPARE, OPEN, USING, CLOSE, DECLARE, END, BEGIN, EXEC, SQL, CONNECT TO, DISCONNECT, COMMIT, LOOP, IF, THEN, ELSE, WHILE, BREAK, EXIT, INSERT, INTO, VALUES, UPDATE, SET, TABLE, PRIMARY, KEY, AND, UNION, ALL, JOIN, ON, GROUP, BY, MATERIALIZED, INT, DATE, COUNT, ORDER, OVER, PARTITION, ASC, DESC, VARCHAR, NOT, NULL, PRIMARY, KEY, DECIMAL, SUM, AVG, ROWS, BETWEEN, PRECEDING, CURRENT, ROW, INSTEAD, TRIGGER, OF, FOR, EACH, EXECUTE, PROCEDURE, DISTINCT, HAVING, LIMIT},
morestring=[s]{'}{'},
morecomment=[l]{--}
        %sensitive=false
}

\lstset{style=mystyle,numbers=none,basicstyle=\ttfamily,upquote=true}

 
\definecolor{codegreen}{rgb}{0,0.6,0}
\definecolor{codegray}{rgb}{0.5,0.5,0.5}
\definecolor{codepurple}{rgb}{0.38,0,0.72}
\definecolor{backcolour}{rgb}{0.95,0.95,0.92}
\definecolor{backcolourSingleCode}{rgb}{0.95,0.95,0.92}

\newcommand{\subtitle}{\textbf{Exercise 10}}
\newcommand{\outdate}{15.01.2024}
\newcommand{\duedate}{22.01.2024 12:00 MEZ}
\newcommand{\video}{056}

\begin{document}

\lhead{\begin{tabular}{l}
{\bf Database Systems WS 2023/24}\\
{\bf \subtitle: Distributed \outdate, Due \duedate}\\
{Submitted by Erik Schwede, Suma Keerthi}
\end{tabular}
}
\rhead{}

\question[1]{Implementing Recovery}
Implement the recovery procedures in a mock database environment.
In OLAT a template is provided, which gives a mock database, with log.

Given this log and the database, your implementation has to
\begin{itemize}
  \item Analyze the log to find the loser transactions
  \item Redo all changes (we assume nothing has been written to disk yet)
  \item Undo all changes of the loser transactions (this includes writing CLRs)
\end{itemize}

The log has the same format as shown in the lecture.
Use the database class to simulate redoing/undoing operations using the \verb+execute+ function.
You can simply use the redo/undo operations of the log entries.
The template also contains the lecture recovery example log as test case.

Submit the code and the output of running your program.

\question[1]{Logging and Recovery with Rollback}

Given a DBMS with concurrent transactions $T_1$, $T_2$, and $T_3$. These transactions perform the operations illustrated below.
The data elements $X$ and $X_i$ are located on page $P_X$.

$T_1$ aborts at timestamp 60($a_1$), while $T_2$ successfully commits at timestamp 90.
The page $P_B$ is flushed at timestamp 55 from the DB buffer.
All rollback operations of $T_1$ are completed at timestamp 65 ($r_1$), before $b_3$ is executed.
The system crashes at timestamp 110, leaving $T_3$ incomplete.

During the executions no checkpoints are set.
The recovery is performed by a full REDO.

\begin{enumerate}
  \item Explain in detail the steps that have to be performed to roll back a transaction.

  \item{Execute the transactions as shown in the illustration and add all required rollback operations. Fill out the following table.}

        \begin{itemize}
          \item Which assumptions/rules/principles did you apply for logging?
          \item What are the operations of $T_1$ at timestamp 61/62?

          \item \emph{Log Entry in Log Buffer}, e.g., [\#02,  $T_1$,    $P_A$,    $R(A_1)$,   $U(A_1)$,       1, 0] (Use $R(\dots)$ / $U(\dots)$ as Redo/Undo information).
          \item \emph{Log File:}Insert the LSNs of the log entries in the log buffer, that are written to the log file.
        \end{itemize}

\end{enumerate}

\newpage

\begin{center}
  \renewcommand{\arraystretch}{2.8}

  \begin{table}[H]
    \small
    \begin{tabular}{|l|l|l|l|l|l|}\hline
      Time & Operation    & DB Buffer              & DB Entry              & Log Entry in Log Buffer                                      & Log File               \\  \hline
           &              & (Page, LSN)            & (Page, LSN)           & [LSN, TA, PageID, Redo, Undo, PrevLSN, UndoNxtLSN]           & LSNs                   \\ \hline
      10   & $b_1$        &                        &                       &                     &                        \\ \hline
      20   & $w_1(A)$     &  &                       &      &                        \\ \hline
      30   & $b_2$        &                        &                       &                     &                        \\ \hline
      40   & $w_1(B_1)$   &  &                       &  &                        \\ \hline
      50   & $w_2(C_1)$   &  &                       &  &                        \\ \hline
      55   & flush($P_B$) &                        &  &                                                              &    \\ \hline
      60   & $a_1$        &                        &                       &                &                        \\ \hline
      61   &              &  &                       &   &                        \\ \hline
      62   &              &  &                       &     &                        \\ \hline
      65   & $r_1$        &                        &                       &                  &      \\ \hline
      70   & $b_3$        &                        &                       &                     &                        \\ \hline
      80   & $w_2(B_2)$   &  &                       &  &                        \\ \hline
      90   & $c_2$        &                        &                       &               &     \\ \hline
      100  & $w_3(C_2)$   &  &                       &  &                        \\ \hline
           &              &                        &                       & \textbf{Crash}                                               &                        \\ \hline
    \end{tabular}
  \end{table}
\end{center}

\newpage

\question[1]{Transaction Rollback}

\begin{enumerate}
  \item Suppose that during transaction rollback no log entries are written.
        Explain what problems will/can arise in this case, by introducing a concrete example.
        Hint: Consider a data item updated by an aborted transaction, and then updated by a transaction that commits.

  \item Consider transactions that involve interactions with the real world, like the transaction of withdrawing money from an ATM or the transaction sending dismissal notices via  postal service.
        Discuss the feasibility of transaction rollback in such cases.
        How would the ``critical'' interactive parts (e.g., releasing the money) of the TA be aligned in time, in order to limit the problematic situations as far as possible?

\end{enumerate}

\question[1]{Schedules - Serializability and Classes}

\begin{enumerate}
  \item Which class does the following schedule have? FSR, VSR, or CSR?
        $$s_1 := r_2(b)\ w_2(b)\ c_2\ w_5(a)\ w_5(b)\ r_3(a)\ r_3(d)\ w_1(b)\ r_3(b)\ r_1(c)\ c_1\ r_3(c)\ r_4(c)\ c_3\ c_5\ w_4(c)\ r_4(d)\ w_4(d)\ c_4\ $$

  \item Given the following schedules, does $s_2 \thickapprox_v s_2'$ hold? Either prove that both schedules are view equivalent or find a counter example.

        $$s_2 := w_2(b)\ r_1(b)\ r_2(c)\ r_3(a)\ r_3(b)\ r_3(a)\ w_1(a)\ w_1(c)\ c_1\ r_2(b)\ w_3(c)\ w_2(c)\ c_2\ c_3\ $$
        $$s_2':= w_2(b)\ r_2(c)\ r_2(b)\ w_2(c)\ c_2\ r_1(b)\  w_1(a)\ w_1(c)\ c_1\ r_3(a)\ r_3(b)\ r_3(a)\ w_3(c)\ c_3\ $$

  \item Given the following schedule

        $$s_4 := w_3(a)\ r_2(c)\ r_3(a)\ c_3\ w_2(a)\ r_2(a)\ c_2\ r_4(a)\ w_1(c)\ r_1(c)\ w_4(b)\ c_4\ w_1(b)\ c_1$$

        Create the conflict graph of $s_4$ and discuss if $s_4 \in CSR$.
        If yes, reorder $s_4$ into a serial schedule using the commutativity rules.

\end{enumerate}

\end{document}
