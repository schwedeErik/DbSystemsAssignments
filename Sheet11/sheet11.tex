
\documentclass[a4paper]{article}
\usepackage[utf8]{inputenc}

\usepackage[english,german]{babel} 
\usepackage[utf8]{inputenc}

\usepackage{alltt}
\usepackage{amsmath}
\usepackage{amssymb}
\usepackage{amsthm}
\usepackage{color}
\usepackage{enumitem}
\usepackage{epsfig}
\usepackage{fancyhdr}
\usepackage{float}
\usepackage{framed}
\usepackage{graphicx} 
\usepackage{graphics}
\usepackage{hyperref}
\usepackage{listings}
\usepackage{multirow}
\usepackage{tabularx}
\usepackage{textcomp}
\usepackage{tikz}
\usepackage{url}
\usepackage{vmargin}
\usepackage{xspace}
\usepackage{comment}
\usetikzlibrary{calc,trees,positioning,arrows,chains,shapes.geometric,%
    decorations.pathreplacing,decorations.pathmorphing,shapes,%
    matrix,shapes.symbols,topaths,matrix}
\newcommand{\question}[2][0]{\section{{#2} \hfill ({#1} P.)}}
\setpapersize{A4}
\setmargins{2.5cm}{2.0cm}% % linker & oberer Rand
         {16cm}{22cm}%   % Textbreite und -hoehe
           {48pt}{36pt}%   % Kopfzeilenhoehe und -abstand
           {0pt}{30pt}%    % \footheight (egal) und Fusszeilenabstand

\frenchspacing
\pagestyle{fancy}
\sloppy

\newcommand{\hide}[1]{}

\markright{Kopfzeile}


\tikzset{
>=stealth',
  punktchain/.style={
    rectangle,
    rounded corners,
    % fill=black!10,
    draw=black, very thick,
    text width=10em,
    minimum height=3em,
    text centered,
    on chain},
  line/.style={draw, thick, <-},
  element/.style={
    tape,
    top color=white,
    bottom color=blue!50!black!60!,
    minimum width=8em,
    draw=blue!40!black!90, very thick,
    text width=10em,
    minimum height=3.5em,
    text centered,
    on chain},
  every join/.style={->, thick,shorten >=1pt},
  decoration={brace},
  tuborg/.style={decorate},
  tubnode/.style={midway, right=2pt},
}


\setlength{\parindent}{0pt}
\setlength{\parskip}{5pt}
\fboxsep1.5mm


\lstdefinestyle{mystyle}{
    backgroundcolor=\color{white},   
    commentstyle=\color{codegray},
    keywordstyle=\bf \ttfamily \color{codepurple},
    numberstyle=\tiny\color{codegray},
    stringstyle=\color{codegreen},
    basicstyle=\footnotesize,
    breakatwhitespace=false,         
    breaklines=true,                 
    captionpos=b,                    
    keepspaces=true,                 
    numbers=left,                    
    numbersep=5pt,                  
    showspaces=false,                
    showstringspaces=false,
    showtabs=false,                  
    tabsize=8,
    keepspaces,
    extendedchars=true, 
      upquote=true,
    columns=fixed,
    showstringspaces=false,
    extendedchars=true,
    breaklines=true,
    frame=single,
    showspaces=false,
    showstringspaces=false,
    rulecolor=\color{white},
}

\lstdefinelanguage{sql}[]{}{
        %tag=[s]<>,      % =*: also apply styles within tag, =**: cumulate styles
        morekeywords={sql, VIEW, AS, FROM, SELECT, WHERE, FUNCTION, BOOLEAN, RETURNS, DETERMINISTIC, RETURN, REFERENCES, WITH, SEQUENCE, TRUNCATE, START, CREATE, AS, LANGUAGE, FUNCTION, CURSOR, PREPARE, OPEN, USING, CLOSE, DECLARE, END, BEGIN, EXEC, SQL, CONNECT TO, DISCONNECT, COMMIT, LOOP, IF, THEN, ELSE, WHILE, BREAK, EXIT, INSERT, INTO, VALUES, UPDATE, SET, TABLE, PRIMARY, KEY, AND, UNION, ALL, JOIN, ON, GROUP, BY, MATERIALIZED, INT, DATE, COUNT, ORDER, OVER, PARTITION, ASC, DESC, VARCHAR, NOT, NULL, PRIMARY, KEY, DECIMAL, SUM, AVG, ROWS, BETWEEN, PRECEDING, CURRENT, ROW, INSTEAD, TRIGGER, OF, FOR, EACH, EXECUTE, PROCEDURE, DISTINCT, HAVING, LIMIT},
morestring=[s]{'}{'},
morecomment=[l]{--}
        %sensitive=false
}

\lstset{style=mystyle,numbers=none,basicstyle=\ttfamily,upquote=true}

 
\definecolor{codegreen}{rgb}{0,0.6,0}
\definecolor{codegray}{rgb}{0.5,0.5,0.5}
\definecolor{codepurple}{rgb}{0.38,0,0.72}
\definecolor{backcolour}{rgb}{0.95,0.95,0.92}
\definecolor{backcolourSingleCode}{rgb}{0.95,0.95,0.92}

\newcommand{\subtitle}{\textbf{Exercise 11}}
\newcommand{\outdate}{22.01.2024}
\newcommand{\duedate}{29.01.2024 12:00 MEZ}
\newcommand{\video}{061}

\begin{document}

\lhead{\begin{tabular}{l}
{\bf Database Systems WS 2023/24}\\
{\bf \subtitle: Distributed \outdate, Due \duedate}\\
{Submitted by Erik Schwede, Suma Keerthi}
\end{tabular}
}
\rhead{}

\question[1]{Schedules - RC, ACA, Strict}

For the following schedules, decide in which of the classes RC, ACA, or ST they are.

\begin{eqnarray*}
s_1\ &:=&r_1(a)\ w_1(a)\ r_2(a)\ r_2(b)\ r_2(a)\ r_1(b)\ c_1\ w_2(b)\ c_2\\
s_2\ &:=&w_1(b)\ r_1(a)\ r_2(a)\ r_1(b)\ w_2(a)\ w_1(b)\ w_2(b)\ c_2\ c_1\\
s_3\ &:=&r_2(a)\ r_1(b)\ r_1(a)\ w_2(a)\ w_1(b)\ w_3(b)\ r_3(b)\ c_3\ c_1\ r_2(b)\ c_2\\
\end{eqnarray*}

\question[1]{2-Phase-Locking and Waits-for-Graph}

\begin{enumerate}
\item
  Given the following schedules:
  \begin{eqnarray*}
  s_1 &=& r_1(c)\ w_2(a)\ r_1(a)\ r_3(b)\ w_3(a)\ c_3\ w_2(c)\ w_1(c)\ c_1\ c_2\\
  s_2 &=& w_1(b)\ r_2(c)\ w_3(a)\ w_2(b)\ r_1(a)\ c_1\ c_3\ w_2(a)\ c_2\ 
  \end{eqnarray*}

  Create a 2PL history for both schedules and note the Waits-for-Graph (WfG) every time it changes.
  If you encounter a deadlock, abort the transaction that caused the deadlock (No deadlock prevention strategy).

\item Given the following WfG.
Which nodes are chosen for the strategies \textbf{most cycles} and \textbf{most edges}.

 \scalebox{0.8}{
 \begin{tikzpicture}[y=.7cm, x=.7cm,font=\sffamily]

   \node[] at (0,0) (t1) {$t_1$}; 
   \node[] at (2,0) (t2) {$t_2$}; 
   \node[] at (-1,1.5) (t3) {$t_3$}; 
   \node[] at (1,1.5) (t4) {$t_4$}; 
   \node[] at (-1,-1.5) (t5) {$t_5$}; 
   \node[] at (1,-1.5) (t6) {$t_6$}; 
   \node[] at (-2,0) (t7) {$t_7$}; 

   \draw[->] (t1) -- (t3);
   \draw[->] (t1) -- (t7);
   \draw[->] (t2) to[bend right] (t4);
   \draw[->] (t3) -- (t4);
   \draw[->] (t4) -- (t1);
   \draw[->] (t4) to[bend right] (t2);
   \draw[->] (t5) -- (t1);
   \draw[->] (t6) -- (t1);
   \draw[->] (t6) -- (t5);

 \end{tikzpicture}
 }

\end{enumerate}

\question[1]{Deadlock prevention}
Implement a scheduler, which creates a (SS)2PL schedule, using the deadlock prevention strategies wait-die and immediate restart.

Test your program with the given schedule.
It has to print the output schedule for each of the prevention strategies and indicate when a transaction has to wait or abort.
After a transaction waits or aborts, you do not have to continue/restart it in the solution history.

$$s :=w_1(x)\ r_2(x)\ w_3(y)\ r_1(y)\ r_3(z)\ w_1(x)\ c_1\ w_2(y)\ c_2\ w_3(y)\ c_3\ $$

You can use the template provided in OLAT, which implements an internal representation of a history and all required operations.

\textbf{Submit the code as separate file and submit the output of running your program with the solution PDF.}

\newpage
\question[1]{Timestamp-Based Approaches}

\begin{align*}
s_{1} &= w_1(x) \ r_2(x) \ w_2(x) \ r_1(x) &\\
s_{2} &= r_1(y) \ w_1(x)  \ r_2(x) \ w_3(x) \ w_1(y) \ w_2(x) \ r_1(x)
\end{align*}

Describe how a timestamp ordering (TO) based scheduler would execute the operations.
Complete the following tables.
Note the used rule in the ``Comment'' column.
The max-$r$ and max-$w$ columns contain the value of max-$q$-scheduled for the given object.
One row contains the entry after applying the rule.
You may assume $ts(t_1) < ts(t_2)$.
Do not restart aborted transactions.

\ \\

$s_1$:
\begin{table}[h]
\centering
\begin{tabular}{|l|c|c|c|c|l|}
\hline
Operation & max-$r$(x) & max-$w$(x) & Comment \\
\hline\hline
$BOT_1$ & $-\infty$ & $-\infty$ &  $ts(t_1) = 1$ \\ \hline
$BOT_2$ & $-\infty$ & $-\infty$ &  $ts(t_2) = 2$ \\ \hline
& & &  \\[5pt]\hline
& & &  \\[5pt]\hline
& & &  \\[5pt]\hline
& & &  \\[5pt]\hline
\end{tabular}
\end{table}

\ \\

$s_2$ \underline{without} Thomas' write rule:
\begin{table}[H]
\centering
\begin{tabular}{|l|c|c|c|c|l|}
\hline
Operation & max-$r$(x) & max-$w$(x) & max-$r$(y) & max-$w$(y) &  Comment \\
\hline\hline
$BOT_1$ & $-\infty$ & $-\infty$ & $-\infty$ & $-\infty$ & $ts(t_1) = 1$ \\ \hline
$BOT_2$ & $-\infty$ & $-\infty$ & $-\infty$ & $-\infty$ & $ts(t_2) = 2$ \\ \hline
$BOT_3$ & $-\infty$ & $-\infty$ & $-\infty$ & $-\infty$ & $ts(t_3) = 3$ \\ \hline
& & & & &   \\[5pt]\hline
& & & & &   \\[5pt]\hline
& & & & &   \\[5pt]\hline
& & & & &   \\[5pt]\hline
& & & & &   \\[5pt]\hline
& & & & &   \\[5pt]\hline
\end{tabular}
\end{table}
\newpage
$s_2$ \underline{with} Thomas' write rule:
\begin{table}[H]
\centering
\begin{tabular}{|l|c|c|c|c|l|}
\hline
Operation & max-$r$(x) & max-$w$(x) & max-$r$(y) & max-$w$(y) &  Comment \\
\hline\hline
$BOT_1$ & $-\infty$ & $-\infty$ & $-\infty$ & $-\infty$ & $ts(t_1) = 1$ \\ \hline
$BOT_2$ & $-\infty$ & $-\infty$ & $-\infty$ & $-\infty$ & $ts(t_2) = 2$ \\ \hline
$BOT_3$ & $-\infty$ & $-\infty$ & $-\infty$ & $-\infty$ & $ts(t_3) = 3$ \\ \hline
& & & & &   \\[5pt]\hline
& & & & &   \\[5pt]\hline
& & & & &   \\[5pt]\hline
& & & & &   \\[5pt]\hline
& & & & &   \\[5pt]\hline
& & & & &   \\[5pt]\hline
& & & & &   \\[5pt]\hline
\end{tabular}
\end{table}

\end{document}
