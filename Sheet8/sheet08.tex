
\documentclass[a4paper]{article}
\usepackage[utf8]{inputenc}

\usepackage[english,german]{babel} 
\usepackage[utf8]{inputenc}

\usepackage{alltt}
\usepackage{amsmath}
\usepackage{amssymb}
\usepackage{amsthm}
\usepackage{color}
\usepackage{enumitem}
\usepackage{epsfig}
\usepackage{fancyhdr}
\usepackage{float}
\usepackage{framed}
\usepackage{graphicx} 
\usepackage{graphics}
\usepackage{hyperref}
\usepackage{listings}
\usepackage{multirow}
\usepackage{tabularx}
\usepackage{textcomp}
\usepackage{tikz}
\usepackage{url}
\usepackage{vmargin}
\usepackage{xspace}
\usepackage{comment}
\usetikzlibrary{calc,trees,positioning,arrows,chains,shapes.geometric,%
    decorations.pathreplacing,decorations.pathmorphing,shapes,%
    matrix,shapes.symbols,topaths,matrix}
\newcommand{\question}[2][0]{\section{{#2} \hfill ({#1} P.)}}
\setpapersize{A4}
\setmargins{2.5cm}{2.0cm}% % linker & oberer Rand
         {16cm}{22cm}%   % Textbreite und -hoehe
           {48pt}{36pt}%   % Kopfzeilenhoehe und -abstand
           {0pt}{30pt}%    % \footheight (egal) und Fusszeilenabstand

\frenchspacing
\pagestyle{fancy}
\sloppy

\newcommand{\hide}[1]{}

\markright{Kopfzeile}


\tikzset{
>=stealth',
  punktchain/.style={
    rectangle,
    rounded corners,
    % fill=black!10,
    draw=black, very thick,
    text width=10em,
    minimum height=3em,
    text centered,
    on chain},
  line/.style={draw, thick, <-},
  element/.style={
    tape,
    top color=white,
    bottom color=blue!50!black!60!,
    minimum width=8em,
    draw=blue!40!black!90, very thick,
    text width=10em,
    minimum height=3.5em,
    text centered,
    on chain},
  every join/.style={->, thick,shorten >=1pt},
  decoration={brace},
  tuborg/.style={decorate},
  tubnode/.style={midway, right=2pt},
}


\setlength{\parindent}{0pt}
\setlength{\parskip}{5pt}
\fboxsep1.5mm


\lstdefinestyle{mystyle}{
    backgroundcolor=\color{white},   
    commentstyle=\color{codegray},
    keywordstyle=\bf \ttfamily \color{codepurple},
    numberstyle=\tiny\color{codegray},
    stringstyle=\color{codegreen},
    basicstyle=\footnotesize,
    breakatwhitespace=false,         
    breaklines=true,                 
    captionpos=b,                    
    keepspaces=true,                 
    numbers=left,                    
    numbersep=5pt,                  
    showspaces=false,                
    showstringspaces=false,
    showtabs=false,                  
    tabsize=8,
    keepspaces,
    extendedchars=true, 
      upquote=true,
    columns=fixed,
    showstringspaces=false,
    extendedchars=true,
    breaklines=true,
    frame=single,
    showspaces=false,
    showstringspaces=false,
    rulecolor=\color{white},
}

\lstdefinelanguage{sql}[]{}{
        %tag=[s]<>,      % =*: also apply styles within tag, =**: cumulate styles
        morekeywords={sql, VIEW, AS, FROM, SELECT, WHERE, FUNCTION, BOOLEAN, RETURNS, DETERMINISTIC, RETURN, REFERENCES, WITH, SEQUENCE, TRUNCATE, START, CREATE, AS, LANGUAGE, FUNCTION, CURSOR, PREPARE, OPEN, USING, CLOSE, DECLARE, END, BEGIN, EXEC, SQL, CONNECT TO, DISCONNECT, COMMIT, LOOP, IF, THEN, ELSE, WHILE, BREAK, EXIT, INSERT, INTO, VALUES, UPDATE, SET, TABLE, PRIMARY, KEY, AND, UNION, ALL, JOIN, ON, GROUP, BY, MATERIALIZED, INT, DATE, COUNT, ORDER, OVER, PARTITION, ASC, DESC, VARCHAR, NOT, NULL, PRIMARY, KEY, DECIMAL, SUM, AVG, ROWS, BETWEEN, PRECEDING, CURRENT, ROW, INSTEAD, TRIGGER, OF, FOR, EACH, EXECUTE, PROCEDURE, DISTINCT, HAVING, LIMIT},
morestring=[s]{'}{'},
morecomment=[l]{--}
        %sensitive=false
}

\lstset{style=mystyle,numbers=none,basicstyle=\ttfamily,upquote=true}

 
\definecolor{codegreen}{rgb}{0,0.6,0}
\definecolor{codegray}{rgb}{0.5,0.5,0.5}
\definecolor{codepurple}{rgb}{0.38,0,0.72}
\definecolor{backcolour}{rgb}{0.95,0.95,0.92}
\definecolor{backcolourSingleCode}{rgb}{0.95,0.95,0.92}

\newcommand{\subtitle}{\textbf{Exercise 8}}
\newcommand{\outdate}{11.12.2023}
\newcommand{\duedate}{01.01.2024 12:00 MEZ}
\newcommand{\video}{040}

\lstset{
    escapeinside={(*}{*)}
}

\begin{document}

\lhead{\begin{tabular}{l}
{\bf Database Systems WS 2023/24}\\
{\bf \subtitle: Distributed \outdate, Due \duedate}\\
{Submitted by Erik Schwede, Suma Keerthi}
\end{tabular}
}
\rhead{}

\begin{center}
\textsf{We wish you happy holidays and a wonderful new year.}
\end{center}

\question[1]{Processing k-NN Queries in  M-Trees}

K Nearest Neighbors (k-NN) query $Q=(q, k)$ returns the $k$ closest points (points that have the shortest distance) to the query point $q$.

\begin{enumerate}
\item Provide a pseudo code for k-NN queries in an M-Tree.

\item Did you apply or can you think of pruning techniques that could be applied to reduce the number of points examined as candidates for the result? Describe why the conditions hold.
\\\\{\bf Solutions:} 
\\\\
The following pseudo code is derived from the paper \dq M-tree: An Efficient Access Method for Similarity Search in Metric Spaces\dq . \href[]{https://www.vldb.org/conf/1997/P426.PDF}{https://www.vldb.org/conf/1997/P426.PDF}\\\\
Similar to the R-Tree also the lower bound and upper bound of the minimal distance are important for the k-NN-query. The algorithm uses a queue of nodes that should the tested. This queue is called PR. The k-NN Objects are stored in an array called NN together with there distance to the query object.
\begin{lstlisting}[escapeinside={(*}{*)}]
PR.add([Root node,NULL])
for(i=0;i<k;i++)
  NN[i] = [NULL,(*$\infty$*)]
while(PR is not empty)
  NextNode = the node with the smallest min distance in the PR 
  knnNodeSearch(NextNode,Q,K)
\end{lstlisting}

Here the code for the knnNodeSearch(NextNode, Q, K) function:\\
Some clarifications:\\
$O_p =$ parent Object of $N$.\\
$d_k =$ distance of the object in the result array with the biggest distance to the query object Q\\
$Q_r =$ Object of $N$.\\
$O_j =$ Object of the leave node.\\
$d_{min}(T(O_r)) =$ The lower bound of the shortest distance between query object Q and $Q_r$\\
$d_{min}(T(O_r)) = max\{d(Q_r,Q)-r(O_r),0\}$\\
$d_{max}(T(O_r)) =$ The upper bound of the shortest distance between query object Q and $Q_r$\\
$d_{max}(T(O_r)) = d(Q_r,Q)+r(O_r)$\\
$r(O_r) =$ the radius of object $O_r$
\begin{lstlisting}[escapeinside={(*}{*)}]
if N is not a leave
  if((*$|d(O_p,Q)-d(O_p,Q_r)|\le d_k+r(O_r)$*))
  calculate (*$d(Q_r,Q)$*)
  if((*$d_{min}(T(O_r))\le d_k$*))
    PR.add([(*$T(O_r)$*), (*$d_{min}(T(O_r))$*)])
    if((*$d_{max}(T(O_r)) < d_k$*))
      Update the NN array with [NULL,(*$d_{max}(T(O_r))$*)]
      Remove all entries from PR that have a higher lower bound 
      for the minimal distance than 
      (*$d_k$*). (*$d_{min}(T(O_r)) > d_k$*)

if N is a leave
  if((*$|d(O_p,Q)-d(O_j,Q_p)|\le d_k$*))
    calculate d(O_j,Q)
    if((*$d(O_j,Q)\le d_k$*))
      Add (*$O_j$*) to the NN Array.
      Remove all elements from PR that have a smaller 
      lower bound minimal distance than the element with
      the biggest distance in the result array NN.
\end{lstlisting}

The solution mentioned above tries to minimize the amount of examined nodes in the tree. In order to achieve this it 
uses the upper and lower bound of the minimal distance between the object of the node and the query object.
It uses the fact that if the lower bound of the minimal distance to an node object is bigger then the upper bound 
of the minimal distance to another node object. The node with the higher lower bound does not have to be examined.
The same strategy is also used in the R-Tree the only difference is the way how to calculate the lower and upper bound of the minimum distance.\\
Example:\\
$d_{max}(T(O_1)) < d_{min}(T(O_2))$\\
$\rightarrow$ The nodes under $T(O_2)$ does not have to be examined anymore. 
\end{enumerate}

\question[1]{Extendible Hashing}  

Implement extendible hashing in a language of your choice. A basic Java template is available in OLAT.

Your program should take a bucket capacity $k$ and a data file as input.
Each line of the file should then be hashed and distributed, using extendible hashing, into buckets with a capacity of $k$.

Build the directory using the prefixes of the hash values.
I.e.: Use the \emph{first} $d$ digits of the hash values.
The buckets are empty in the beginning, thereby, the directory has size $d=0$.

Your implementation must have the following features:
\begin{itemize}
\item Take data file as parameter
\begin{itemize}
  \item Calculate the hash of each line
\end{itemize}
\item Take maximum bucket size as parameter
\item After each insertion step, return the current directory (The buckets, their content, the local depth, the hash prefixes pointing to them, and the global depth)
\end{itemize}

The template already implements reading the file and hashing each line.
Submit the source code and the output of your program, when executed with the data file in OLAT and $k=3$.
\\\\{\bf Solution:}\\\\
For a bucket size of 3 the program produces the following output:
\begin{lstlisting}
Initialized Directory with maximum bucket size: 3
Reading file: data.txt
Inserting 'YwbeN' with hash value 01010000
======================================
Global depth d:0
Data buckets:
  Bucket: C = 0, Prefix = ''
    Data entries:
      'YwbeN'  01010000
      

======================================
Inserting 'uubRH' with hash value 10001101
======================================
Global depth d:0
Data buckets:
  Bucket: C = 0, Prefix = ''
    Data entries:
      'YwbeN'  01010000
      'uubRH'  10001101
      

======================================
Inserting 'ZagSZ' with hash value 01110110
======================================
Global depth d:0
Data buckets:
  Bucket: C = 0, Prefix = ''
    Data entries:
      'YwbeN'  01010000
      'uubRH'  10001101
      'ZagSZ'  01110110
      

======================================
Increasing global depth!
Spliting bucket with prefix: ''
Inserting 'Allp6' with hash value 10000100
======================================
Global depth d:1
Data buckets:
  Bucket: C = 1, Prefix = '0'
    Data entries:
      'YwbeN'  01010000
      'ZagSZ'  01110110
      
  Bucket: C = 1, Prefix = '1'
    Data entries:
      'uubRH'  10001101
      'Allp6'  10000100
      

======================================
Inserting '1cDee' with hash value 11000001
======================================
Global depth d:1
Data buckets:
  Bucket: C = 1, Prefix = '0'
    Data entries:
      'YwbeN'  01010000
      'ZagSZ'  01110110
      
  Bucket: C = 1, Prefix = '1'
    Data entries:
      'uubRH'  10001101
      'Allp6'  10000100
      '1cDee'  11000001
      

======================================
Increasing global depth!
Spliting bucket with prefix: '1'
Inserting 'TTn7K' with hash value 10110111
======================================
Global depth d:2
Data buckets:
  Bucket: C = 2, Prefix = '11'
    Data entries:
      '1cDee'  11000001
      
  Bucket: C = 1, Prefix = '0'
    Data entries:
      'YwbeN'  01010000
      'ZagSZ'  01110110
      
  Bucket: C = 2, Prefix = '10'
    Data entries:
      'uubRH'  10001101
      'Allp6'  10000100
      'TTn7K'  10110111
      

======================================
Inserting 'UvPeA' with hash value 11111110
======================================
Global depth d:2
Data buckets:
  Bucket: C = 2, Prefix = '11'
    Data entries:
      '1cDee'  11000001
      'UvPeA'  11111110
      
  Bucket: C = 1, Prefix = '0'
    Data entries:
      'YwbeN'  01010000
      'ZagSZ'  01110110
      
  Bucket: C = 2, Prefix = '10'
    Data entries:
      'uubRH'  10001101
      'Allp6'  10000100
      'TTn7K'  10110111
      

======================================
Inserting 'JNhlA' with hash value 00001001
======================================
Global depth d:2
Data buckets:
  Bucket: C = 2, Prefix = '11'
    Data entries:
      '1cDee'  11000001
      'UvPeA'  11111110
      
  Bucket: C = 1, Prefix = '0'
    Data entries:
      'YwbeN'  01010000
      'ZagSZ'  01110110
      'JNhlA'  00001001
      
  Bucket: C = 2, Prefix = '10'
    Data entries:
      'uubRH'  10001101
      'Allp6'  10000100
      'TTn7K'  10110111
      

======================================
Spliting bucket with prefix: '0'
Inserting '2QU1C' with hash value 00100000
======================================
Global depth d:2
Data buckets:
  Bucket: C = 2, Prefix = '01'
    Data entries:
      'YwbeN'  01010000
      'ZagSZ'  01110110
      
  Bucket: C = 2, Prefix = '11'
    Data entries:
      '1cDee'  11000001
      'UvPeA'  11111110
      
  Bucket: C = 2, Prefix = '00'
    Data entries:
      'JNhlA'  00001001
      '2QU1C'  00100000
      
  Bucket: C = 2, Prefix = '10'
    Data entries:
      'uubRH'  10001101
      'Allp6'  10000100
      'TTn7K'  10110111
      

======================================
Inserting 'XLvlv' with hash value 01110110
======================================
Global depth d:2
Data buckets:
  Bucket: C = 2, Prefix = '01'
    Data entries:
      'YwbeN'  01010000
      'ZagSZ'  01110110
      'XLvlv'  01110110
      
  Bucket: C = 2, Prefix = '11'
    Data entries:
      '1cDee'  11000001
      'UvPeA'  11111110
      
  Bucket: C = 2, Prefix = '00'
    Data entries:
      'JNhlA'  00001001
      '2QU1C'  00100000
      
  Bucket: C = 2, Prefix = '10'
    Data entries:
      'uubRH'  10001101
      'Allp6'  10000100
      'TTn7K'  10110111
      

======================================

Process finished with exit code 0

\end{lstlisting}

{\bf Observation:}\\
The algorithm does not work when the minimum bucket size is lower then the amount of data values that have the exact same hash value. For that scenario there must be overflow buckets.\\
Example:\\
Maximum bucket size = 1\\
Value: 'XLvlv' Hash value: 01110110\\
Value: 'ZagSZ' Hash value: 01110110\\
It does not mater how often the buckets get split they will always fall into the same bucket which is not possible because of the maximum bucket size of 1.

\newpage
\question[1]{Linear Hashing}

Perform linear hashing for the following given parameters:

Using the following sequence of hash functions:
\[
H_i(K) = K \mod (2 \cdot 2^i) \textnormal{ with } i \in \{0, 1, 2, \dots, n\}
\]

The hash table should be initialized with 2 buckets.
Each bucket has a capacity of 3 entries.
If more than $\beta > \frac{2}{3}$ of the table is occupied, controlled splitting should be performed.

Insert the following values in the given order:

\[
27, 13, 28, 3, 21, 8, 27, 16, 36
\]

Write down what happens during each insert.
Also visualize your buckets after every split.\\\\
{\bf Solutions:}\\\\
Number of buckets to be considered ,\textbf{N = 2}

Number of records per bucket to be considered, \textbf{b=3}

Given \textbf{hash function} to be considered 

    \[H_i(K) = K \mod (2 \cdot 2^i) with i \in \{0, 1, 2, \dots, n\}\]

Given threshold = $\beta_s = 2/3 = 0.66$

Number of values inserted into buckets, \textbf{x = 8}

As we are using Linear hashing, we would have a pointer\textbf{ p}, pointing to the first bucket initially and follows the round robin fashion to split the buckets.
After applying the hash function, the output of the hash function would be the bucket numbers the values would be placed into.
\[\textbf{Values: 27, 13, 28, 3, 21, 8, 27, 16, 36}\]
\begin{center}
  \textbf{(i) Inserting 27 in the bucket and i=0}
  \[H_0(27) = 27 mod (2 \cdot 2^0)\]
  \[H_0(27) = 27 mod (2 \cdot 1) = 27 mod 2 = 1\] 
    
\end{center}

\begin{center}
\begin{tabular}{ |c|c| } 
 \hline
  \textbf{\color{red}{P}} BUCKET 0 & BUCKET 1 \\ [0.5ex] 
 \hline\hline
 \hline
   & 27 \\ 
   &    \\
   &    \\
 \hline
\end{tabular}
\end{center}
\textbf{(ii) Inserting 13 in the bucket}
\[H_0(13) = 13 mod (2 \cdot 1)  = 13 mod 2 = 1\] 

\begin{center}
\begin{tabular}{ |c|c| } 
 \hline
 \textbf{\color{red}{P}} BUCKET 0 & BUCKET 1 \\ [0.5ex] 
 \hline\hline
 \hline
   & 27 \\ 
   & 13 \\
   &    \\
   \hline
\end{tabular}
\end{center}

\newpage

\textbf{(iii) Inserting 28 in the bucket}
\[H_0(28) = 28 mod (2 \cdot 1)  = 28 mod 2 = 0\] 

\begin{center}
\begin{tabular}{ |c|c| } 
 \hline
 \textbf{\color{red}{P}} BUCKET 0 & BUCKET 1 \\ [0.5ex] 
 \hline\hline
 \hline
  28 & 27 \\ 
   & 13   \\
   &      \\
 \hline
\end{tabular}
\end{center}

\textbf{(iv) Inserting 3 in the bucket}
\[H_0(3) = 3 mod (2 \cdot 1)  = 3 mod 2 = 1\] 

\begin{center}
\begin{tabular}{ |c|c| } 
 \hline
 \textbf{\color{red}{P}} BUCKET 0 & BUCKET 1 \\ [0.5ex] 
 \hline\hline
 \hline
  28 & 27 \\ 
   & 13   \\
   & 3 \\
 \hline
\end{tabular}
\end{center}

\textbf{(v) Inserting 21 in the bucket}
\[H_0(21) = 21 mod (2 \cdot 1)  = 21 mod 2 = 1\] 

\begin{center}
\begin{tabular}{ |c|c| } 
 \hline
 \textbf{\color{red}{P}} BUCKET 0 & BUCKET 1 \\ [0.5ex] 
 \hline\hline
 \hline
  28 & 27 \\ 
   & 13   \\
   & 3 \\
 \hline
    & \textbf{\textcolor{blue}{21}}
    
\end{tabular}
\end{center}

The bucket 1 is full and hence the value of 21 goes to the overflow bucket. 
We have to calculate the value of $\beta$

\[\beta = x/(b \times M)\]
\[\beta = 5/(2 \times 3) = 5/6 = 0.83\]

But $\beta_s = 0.66 , \beta>\beta_s$,\textbf{ hence the controlled splitting must be performed}. The pointer points to bucket 0, and hence we \textbf{split the bucket 0 into bucket 0 and bucket 2.} Bucket 0 and Bucket 2 follow $H_1$ and bucket 1 follows $H_0$
\\
\\
\begin{center}
\begin{tabular}{ |c|c|c| } 
 \hline
 BUCKET 0 & \textbf{\color{red}{P}} BUCKET 1 & BUCKET 2\\ [0.5ex] 
 \hline\hline
 \hline
  28 & 27 & \\ 
   & 13 & \\
   & 3 & \\
 \hline
 \hline
\textbf{\textcolor{magenta}{h1}} & \textbf{\textcolor{teal}{h0}}& \textbf{\textcolor{magenta}{h1}}\\ [0.5ex] 
 \hline
     & \textbf{\textcolor{blue}{21}}
\end{tabular}
\end{center}
\newpage

Apply $H_1$ function to the values that are already present in \textbf{BUCKET 0}
\[H_1(28) = 28 mod (2 \cdot 2.2^1)  = 28 mod 4 = 0\]

After $H_1$ function, the value of 28 remains in bucket 0.


\textbf{(vi) Inserting 8 in the bucket}
\[H_0(8) = 8 mod (2 \cdot 1)  = 8 mod 2 = 0\]
    p is the pointer pointing to bucket 1, and therefore the value of \textbf{p=1}
\[H_0(8)\ge p = False\]

\[H_0(8) < p = 0 < 1 = True\]

Hence use $H_1$ function to place the value of 8 in bucket
\[H_1(8) = 8 mod (2 \cdot 2.2^1)  = 8 mod 4 = 0\]


\begin{center}
\begin{tabular}{ |c|c|c| } 
 \hline
 BUCKET 0 & \textbf{\color{red}{P}} BUCKET 1 & BUCKET 2\\ [0.5ex] 
 \hline\hline
 \hline
  28 & 27 & \\ 
   8 & 13 & \\
   & 3 & \\
 \hline
 \hline
\textbf{\textcolor{magenta}{h1}} & \textbf{\textcolor{teal}{h0}}& \textbf{\textcolor{magenta}{h1}}\\ [0.5ex] 
 \hline
     & \textbf{\textcolor{blue}{21}}
\end{tabular}
\end{center}
\textbf{(vii) Inserting 27 in the bucket}

For the upcoming values, we don't know if it belongs to $H_0$ or $H_1$ and hence we apply ${H_0}$ first.

\[H_0(27) = 27 mod (2 \cdot 1)  = 27 mod 2 = 1\] 

The bucket 1 is full and hence the value of 27 goes to the overflow bucket.

\begin{center}
\begin{tabular}{ |c|c|c| } 
 \hline
 BUCKET 0 & \textbf{\color{red}{P}} BUCKET 1 & BUCKET 2\\ [0.5ex] 
 \hline\hline
 \hline
  28 & 27 & \\ 
   8 & 13 & \\
   & 3 & \\
 \hline
 \hline
\textbf{\textcolor{magenta}{h1}} & \textbf{\textcolor{teal}{h0}}& \textbf{\textcolor{magenta}{h1}}\\ [0.5ex] 
 \hline
     & \textbf{\textcolor{blue}{21}}\\ 
     & \textbf{\textcolor{blue}{27}}
\end{tabular}
\end{center}

We have to calculate the value of $\beta$

\[\beta = x/(b \times M)\]
\[\beta = 7/(3 \times 3) = 7/9 = 0.77\]

But $\beta_s = 0.66 , \beta>\beta_s$,\textbf{ hence the controlled splitting must be performed}. The pointer points to bucket 1, and hence we \textbf{split the bucket 1 into bucket 1 and bucket 3.}Now that all the buckets are split doubly, the pointer moves back to bucket 0 as per round robin fashion and \textbf{all the buckets follow $H_1$ function} and \textbf{$H_0$ function is eliminated.}


\newpage

$H_1$ function is applied to all the values that are present previously in the table and the values get reshuffled due to this.

\[H_1(27) = 27 mod (2 \cdot 2.2^1)  = 27 mod 4 = 3\]
\[H_1(13) = 13 mod (2 \cdot 2.2^1)  = 13 mod 4 = 1\]
\[H_1(28) = 28 mod (2 \cdot 2.2^1)  = 28 mod 4 = 0\]
\[H_1(3) = 3 mod (2 \cdot 2.2^1)  = 3 mod 4 = 3\]
\[H_1(21) = 21 mod (2 \cdot 2.2^1)  = 21 mod 4 = 1\]
\[H_1(8) = 8 mod (2 \cdot 2.2^1)  = 8 mod 4 = 0\]
\[H_1(27) = 27 mod (2 \cdot 2.2^1)  = 27 mod 4 = 3\]
\begin{center}
\begin{tabular}{ |c|c|c|c| } 
 \hline
 \textbf{\color{red}{P}} BUCKET 0 & BUCKET 1 & BUCKET 2 & BUCKET 3\\ [0.5ex] 
 \hline\hline
 \hline
  28 & 13 &  &27\\ 
   4 & 21 &  &3\\
    &  &  &\\
 \hline
 \hline
\textbf{\textcolor{magenta}{h1}} & \textbf{\textcolor{magenta}{h1}}& \textbf{\textcolor{magenta}{h1}}&\textbf{\textcolor{magenta}{h1}}\\ [0.5ex] 
 \hline
\end{tabular}
\end{center}
\textbf{(viii) Inserting 16 in the bucket}
\[H_1(16) = 16 mod (2 \cdot 2.2^1)  = 16 mod 4 = 0\]
\begin{center}
\begin{tabular}{ |c|c|c|c| } 
 \hline
 \textbf{\color{red}{P}} BUCKET 0 & BUCKET 1 & BUCKET 2 & BUCKET 3\\ [0.5ex] 
 \hline\hline
 \hline
  28 & 13 &  &27\\ 
   4 & 21 &  &3\\
    16 &  &  &\\
 \hline
 \hline
\textbf{\textcolor{magenta}{h1}} & \textbf{\textcolor{magenta}{h1}}& \textbf{\textcolor{magenta}{h1}}&\textbf{\textcolor{magenta}{h1}}\\ [0.5ex] 
 \hline
\end{tabular}
\end{center}
\textbf{(ix) Inserting 36 in the bucket}
\[H_1(36) = 36 mod (2 \cdot 2.2^1)  = 36 mod 4 = 0\]

The bucket 0 is full and hence the value of 27 goes to the overflow bucket.
\\
\\
\begin{center}
\begin{tabular}{ |c|c|c|c| } 
 \hline
 \textbf{\color{red}{P}} BUCKET 0 & BUCKET 1 & BUCKET 2 & BUCKET 3\\ [0.5ex] 
 \hline\hline
 \hline
  28 & 13 &  &27\\ 
   4 & 21 &  &3\\
    16 &  &  &\\
 \hline
 \hline
\textbf{\textcolor{magenta}{h1}} & \textbf{\textcolor{magenta}{h1}}& \textbf{\textcolor{magenta}{h1}}&\textbf{\textcolor{magenta}{h1}}\\ [0.5ex] 
 \hline
 \textbf{\textcolor{blue}{36}}
\end{tabular}
\end{center}
We have to calculate the value of $\beta$

\[\beta = x/(b \times M)\]
\[\beta = 9/(4 \times 3) = 9/12 = 0.75\]

But $\beta_s = 0.66 , \beta>\beta_s$,\textbf{ hence the controlled splitting must be performed}. The pointer points to bucket 0, and hence we \textbf{split the bucket 0 into bucket 0 and bucket 4.}Bucket 0 and Bucket 4 follow $H_2$ function and bucket1, bucket 2 and bucket 3 follows $H_1$ function.

Apply $H_2$ for the existing elements of the bucket 0
\\
\\
\begin{center}
\begin{tabular}{ |c|c|c|c|c|} 
 \hline
  BUCKET 0 & \textbf{\color{red}{P}} BUCKET 1 & BUCKET 2 & BUCKET 3 & BUCKET 4\\ [0.5ex] 
 \hline\hline
 \hline
  16 & 13 &  & 27& 28\\ 
    & 21 &  & 3& 4\\
     &  &  &  &\\
 \hline
 \hline
\textbf{\textcolor{cyan}{h2}} & \textbf{\textcolor{magenta}{h1}}& \textbf{\textcolor{magenta}{h1}}&\textbf{\textcolor{magenta}{h1}}&\textbf{\textcolor{cyan}{h2}}\\ [0.5ex] 
 \hline
 \textbf{\textcolor{blue}{36}}
\end{tabular}
\end{center}

Applying $H_2$ function for 36 we get
\[H_2(36) = 36 mod (2 \cdot 2.2^2)  = 36 mod 8 = 4\]

Final output of values in the buckets are as follows
\\
\\
\begin{center}
\begin{tabular}{ |c|c|c|c|c|} 
 \hline
  BUCKET 0 & \textbf{\color{red}{P}} BUCKET 1 & BUCKET 2 & BUCKET 3 & BUCKET 4\\ [0.5ex] 
 \hline\hline
 \hline
  16 & 13 &  & 27& 28\\ 
    & 21 &  & 3& 4\\
     &  &  &  &36\\
 \hline
 \hline
\textbf{\textcolor{cyan}{h2}} & \textbf{\textcolor{magenta}{h1}}& \textbf{\textcolor{magenta}{h1}}&\textbf{\textcolor{magenta}{h1}}&\textbf{\textcolor{cyan}{h2}}\\ [0.5ex] 
 \hline
\end{tabular}
\end{center}
\question[1]{Top-k Algorithms}

Apply the FA and TA algorithm for $k=2$, using addition as aggregation function, on the following three index lists.
 Write down all index list accesses, as well as the current top-$k$ documents after each step.
 How many sequential and how many random accesses were executed?

  \begin{table}[h]
    \begin{center}
      \begin{minipage}[t]{2cm}
        \begin{tabular}{|p{25pt}|}\hline
          $L_1$\\\hline
          $d_2 \, \, 0.9$\\\hline
          $d_3 \, \, 0.8$\\\hline
          $d_1 \, \, 0.5$\\\hline
          $d_6 \, \, 0.4$\\\hline
          $d_5 \, \, 0.3$\\\hline
          $d_8 \, \, 0.2$\\\hline
          $d_7 \, \, 0.1$\\\hline
        \end{tabular}
      \end{minipage}
      \hspace{5mm}
      \begin{minipage}[t]{2cm}
        \begin{tabular}{|p{25pt}|}\hline
          $L_2$\\\hline
          $d_1 \, \, 0.8$\\\hline
          $d_2 \, \, 0.7$\\\hline
          $d_3 \, \, 0.5$\\\hline
          $d_6 \, \, 0.4$\\\hline
          $d_8 \, \, 0.3$\\\hline
          $d_4 \, \, 0.3$\\\hline
          $d_7 \, \, 0.1$\\\hline
        \end{tabular}
      \end{minipage}
      \hspace{5mm}
      \begin{minipage}[t]{2cm}
        \begin{tabular}{|p{25pt}|}\hline
          $L_3$\\\hline
          $d_3 \, \, 0.9$\\\hline
          $d_4 \, \, 0.8$\\\hline
          $d_1 \, \, 0.6$\\\hline
          $d_2 \, \, 0.4$\\\hline
          $d_5 \, \, 0.3$\\\hline
          $d_7 \, \, 0.2$\\\hline
          $d_8 \, \, 0.2$\\\hline
        \end{tabular}
      \end{minipage}
    \end{center}
  \end{table}

  \textbf{Solution:} 
\\
\textbf{(i) Fagin's Algorithm}

For k=2, we do sequential reads until we find all top 2 values as k=2.

Aggregation function: \textbf{sum}

    \begin{center}
      \begin{minipage}[t]{2cm}
        \begin{tabular}{|p{25pt}|}\hline
          $L_1$\\\hline
          $d_2 \, \, 0.9$\\\hline
          $d_3 \, \, 0.8$\\\hline
          $d_1 \, \, 0.5$\\\hline
        \end{tabular}
    \end{minipage}
    \hspace{5mm}
    \begin{minipage}[t]{2cm}
        \begin{tabular}{|p{25pt}|}\hline
          $L_2$\\\hline
          $d_1 \, \, 0.8$\\\hline
          $d_2 \, \, 0.7$\\\hline
          $d_3 \, \, 0.5$\\\hline
        \end{tabular}
    \end{minipage}
    \hspace{5mm}
    \begin{minipage}[t]{2cm}
        \begin{tabular}{|p{25pt}|}\hline
        $L_3$\\\hline
        $d_3 \, \, 0.9$\\\hline
        $d_4 \, \, 0.8$\\\hline
        $d_1 \, \, 0.6$\\\hline
        \end{tabular}
    \end{minipage}
    \end{center}

Calculating the values of the documents

For the pages that are not in the sequential scan, we consider as 0 by default.
\[d1 = 0.5 + 0.8 + 0.6 = 1.9 \] 
\[d2 = 0.9 + 0.7 + 0 = 1.6 \]
\[d3 = 0.8 + 0.5 + 0.9 = 2.2 \]

Top values of k=2 so far are \textbf{d3,d1}
We have performed 3 sequential reads per row in the index list, hence we have performed \textbf{9 sequential reads} on the whole so far.

There are possibilities that the partially scanned documents might have the better values and henc we perform random access on the partially scanned documents.

\[d2 = 0.9 + 0.7 + 0.4 = 2 \]  
\[d1 = 0 + 0.3 + 0.8 = 1.1 \] 

$d_2$ takes 2 random scans, one for L1 and other for L2. d4 takes one random scan, summing up to 3 random scans.
As, $d_2$ value is greater than $d_1$, we consider $d_2$ to be the top element along with $d_3$


Hence, \textbf{{d3,d2}} are the top 2 documents for k=2. 

\textbf{Sequential accesses executed : 9}

\textbf{Random accesses executed : 3}
\\

\textbf{(ii) Threshold Algorithm}:  Read sequentially from each index list and perform random reads for every document found in the sequence list. We compare the threshold of the aggregated values of the documents with the scan line score. When the value of \textbf{aggregated values $>$ scan line score}, then we terminate the algorithm.
\\
\\

\begin{center}
    \begin{minipage}[t]{2cm}
        \begin{tabular}{|p{25pt}|}\hline
          $L_1$\\\hline
          $d_2 \, \, 0.9$\\\hline
        \end{tabular}
    \end{minipage}
    \hspace{5mm}
    \begin{minipage}[t]{2cm}
        \begin{tabular}{|p{25pt}|}\hline
          $L_2$\\\hline
          $d_1 \, \, 0.8$\\\hline
            \end{tabular}
    \end{minipage}
    \hspace{5mm}
    \begin{minipage}[t]{2cm}
        \begin{tabular}{|p{25pt}|}\hline
        $L_3$\\\hline
        $d_3 \, \, 0.9$\\\hline      
        \end{tabular}
    \end{minipage}
    \end{center}
\begin{center}
  \begin{tabular}{ |c|c| } 
    \hline
     \textbf{SEQUENTIAL ACCESS} & \textbf{RANDOM ACCESS}  \\ [0.5ex] 
    \hline\hline
    \textbf{ L1: $d_2$ 0.9} &  \textbf{L2:} $d_2$ 0.7, \textbf{L3:} $d_2$ 0.4 \\ 
     \hline
     \textbf{ L2: $d_1$ 0.8} &  \textbf{L1:} $d_1$ 0.5,\textbf{ L3:} $d_1$ 0.6 \\
     \hline
      \textbf{ L3: $d_3$ 0.9} & \textbf{ L2:} $d_3$ 0.8, \textbf{L3:} $d_3$ 0.5 \\
    \hline
   \end{tabular}
\end{center}  


Aggregation sum of the documents:
\[d2 = 0.9 + 0.7 + 0.4 = 2 \]
\[d1 = 0.8 + 0.5 + 0.6 = 1.9 \] 
\[d3 = 0.9 + 0.8 + 0.5 = 2.2 \]

Scan line score = L1 + L2 + L3 = 0.9 + 0.8 + 0.9 = 2.6

Current top-k: \textbf\{{d3,d2}\}

The aggregated values of\textbf{ d3 and d2 $<$ scan line score} - continue algorithm
\\
\textbf{Sequential access for L1: d3 0.8}
\\
\begin{center}
    \begin{minipage}[t]{2cm}
        \begin{tabular}{|p{25pt}|}\hline
          $L_1$\\\hline
          $d_2 \, \, 0.9$\\\hline
          $d_3 \, \, 0.8$\\\hline
        \end{tabular}
    \end{minipage}
    \hspace{5mm}
    \begin{minipage}[t]{2cm}
        \begin{tabular}{|p{25pt}|}\hline
          $L_2$\\\hline
          $d_1 \, \, 0.8$\\\hline
            \end{tabular}
    \end{minipage}
    \hspace{5mm}
    \begin{minipage}[t]{2cm}
        \begin{tabular}{|p{25pt}|}\hline
        $L_3$\\\hline
        $d_3 \, \, 0.9$\\\hline      
        \end{tabular}
    \end{minipage}
    \end{center}

 Scan line score = 0.8 + 0.8 + 0.9 = 2.5

 The aggregated values of\textbf{ d3 and d2 $<$ scan line score} - continue algorithm
\\
\textbf{Sequential access for L2: d2 0.7}
 \begin{center}
    \begin{minipage}[t]{2cm}
        \begin{tabular}{|p{25pt}|}\hline
          $L_1$\\\hline
          $d_2 \, \, 0.9$\\\hline
          $d_3 \, \, 0.8$\\\hline
        \end{tabular}
    \end{minipage}
    \hspace{5mm}
    \begin{minipage}[t]{2cm}
        \begin{tabular}{|p{25pt}|}\hline
          $L_2$\\\hline
          $d_1 \, \, 0.8$\\\hline
          $d_2 \, \, 0.7$\\\hline
            \end{tabular}
    \end{minipage}
    \hspace{5mm}
    \begin{minipage}[t]{2cm}
        \begin{tabular}{|p{25pt}|}\hline
        $L_3$\\\hline
        $d_3 \, \, 0.9$\\\hline      
        \end{tabular}
    \end{minipage}
    \end{center}

Scan line score = 0.8 + 0.7 + 0.9 = 2.4

The aggregated values of\textbf{ d3 and d2 $<$ scan line score} - continue algorithm
\\
\textbf{Sequential access for L3: d4 0.8}

\textbf{Random access for L1 d4:0 and L2 d4:0.3}

Aggregation of d4 $d4 = 0 + 0.8 + 0.3 = 1.1$
 \begin{center}
    \begin{minipage}[t]{2cm}
        \begin{tabular}{|p{25pt}|}\hline
          $L_1$\\\hline
          $d_2 \, \, 0.9$\\\hline
          $d_3 \, \, 0.8$\\\hline
        \end{tabular}
    \end{minipage}
    \hspace{5mm}
    \begin{minipage}[t]{2cm}
        \begin{tabular}{|p{25pt}|}\hline
          $L_2$\\\hline
          $d_1 \, \, 0.8$\\\hline
          $d_2 \, \, 0.7$\\\hline
            \end{tabular}
    \end{minipage}
    \hspace{5mm}
    \begin{minipage}[t]{2cm}
        \begin{tabular}{|p{25pt}|}\hline
        $L_3$\\\hline
        $d_3 \, \, 0.9$\\\hline 
        $d_4 \, \, 0.8$\\\hline
        \end{tabular}
    \end{minipage}
    \end{center}

Scan line score = 0.8 + 0.7 + 0.8 = 2.3

The aggregated values of\textbf{ d3 and d2 $<$ scan line score} - continue algorithm 
\\
\textbf{Sequential access for L1: d1 0.5}
 \begin{center}
    \begin{minipage}[t]{2cm}
        \begin{tabular}{|p{25pt}|}\hline
          $L_1$\\\hline
          $d_2 \, \, 0.9$\\\hline
          $d_3 \, \, 0.8$\\\hline
          $d_1 \, \, 0.5$\\\hline
        \end{tabular}
    \end{minipage}
    \hspace{5mm}
    \begin{minipage}[t]{2cm}
        \begin{tabular}{|p{25pt}|}\hline
          $L_2$\\\hline
          $d_1 \, \, 0.8$\\\hline
          $d_2 \, \, 0.7$\\\hline
            \end{tabular}
    \end{minipage}
    \hspace{5mm}
    \begin{minipage}[t]{2cm}
        \begin{tabular}{|p{25pt}|}\hline
        $L_3$\\\hline
        $d_3 \, \, 0.9$\\\hline 
        $d_4 \, \, 0.8$\\\hline
        \end{tabular}
    \end{minipage}
    \end{center}
    
Scan line score = 0.5 + 0.7 + 0.8 = 2

The aggregated values of\textbf{ d3 and d2 $<$ scan line score} - continue algorithm 

\textbf{Sequential access for L2: d1 0.5}
 \begin{center}
    \begin{minipage}[t]{2cm}
        \begin{tabular}{|p{25pt}|}\hline
          $L_1$\\\hline
          $d_2 \, \, 0.9$\\\hline
          $d_3 \, \, 0.8$\\\hline
          $d_1 \, \, 0.5$\\\hline
        \end{tabular}
    \end{minipage}
    \hspace{5mm}
    \begin{minipage}[t]{2cm}
        \begin{tabular}{|p{25pt}|}\hline
          $L_2$\\\hline
          $d_1 \, \, 0.8$\\\hline
          $d_2 \, \, 0.7$\\\hline
          $d_3 \, \, 0.5$\\\hline
            \end{tabular}
    \end{minipage}
    \hspace{5mm}
    \begin{minipage}[t]{2cm}
        \begin{tabular}{|p{25pt}|}\hline
        $L_3$\\\hline
        $d_3 \, \, 0.9$\\\hline 
        $d_4 \, \, 0.8$\\\hline
        \end{tabular}
    \end{minipage}
    \end{center}
    
Scan line score = 0.5 + 0.5 + 0.8 = 1.8

The aggregated values of\textbf{ d3 and d2 $>$ scan line score} - \textbf{Terminate algorithm }  

\textbf{Sequential Accesses: 8
Random Accesses : 8}

\end{document}
